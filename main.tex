    \documentclass{mipt-thesis-bs}
    \usepackage{mipt-thesis-biblatex}
    \addbibresource{bibliography.bib}
    \usepackage{hyperref}
    %\urlstyle{same}

    \title{Влияние микролинзирования на кривые
           блеска гравитационно линзированных сверхновых}
    \author{Круглов А.\ А.}
    \supervisor{Лыскова Н.\ С.}
    \groupnum{082}
    \faculty{Факультет проблем физики и энергетики}
    \department{Кафедра космической физики}

\begin{document}

    \frontmatter
    
    \titlecontents
    
    \mainmatter


\chapter{Введение}

    \section{Актуальность}
    Здесь идет текст. Вот так выглядит ссылка на библиографию \cite{adams1995hitchhiker}.
    
    Заявка на грант: \href{https://docs.google.com/document/d/1LeG-XjcpTT6cA9TJIDq61qVxFLF8IR38VrDoK6hkXco/edit?ts=5b23a829#}{link}.
    \section{Цели и задачи научной работы}
    Аналогично добавляются \cite[chapter 2]{langmuir26} еще главы, внутри них можно объявлять секции с помощью \verb|\section|.


    \backmatter

    \printbib
   
    \end{document}
