    \documentclass{mipt-thesis-bs}
    \usepackage{mipt-thesis-biblatex}
    \addbibresource{bibliography.bib}
    \usepackage{hyperref}
    %\urlstyle{same}

    \title{Влияние микролинзирования на кривые
           блеска гравитационно линзированных сверхновых}
    \author{Круглов А.\ А.}
    \supervisor{Лыскова Н.\ С.}
    \groupnum{582}
    \faculty{Факультет проблем физики и энергетики}
    \department{Кафедра космической физики}

\begin{document}

    \frontmatter
    
    \titlecontents
    
    \mainmatter


\chapter{Введение}

    \section{Тест литературы}
    \cite{gravlensbook}
    \cite{kelly2014}
    \cite{treu2015}
    \cite{rodney2016}
    \cite{gl_all}
    \cite{moresuyu2017}
    \cite{pierelrodney2019}
    \cite{dobler2015}
    \cite{doblerkeeton2006}
    \cite{refsdalstabell1991}
    \cite{refsdal1964}
    \cite{narbart}
    
    \section{Актуальность задачи}
    Заявка на грант: \href{https://docs.google.com/document/d/1LeG-XjcpTT6cA9TJIDq61qVxFLF8IR38VrDoK6hkXco/edit?ts=5b23a829#}{link}.
    
    \section{Цели и задачи научной работы}
    Аналогично добавляются \cite{langmuir26} еще главы \cite{adams1995hitchhiker}, внутри них можно объявлять секции с помощью \verb|\section|.
    
    \section{Собственно введение}
    Гравитационное линзирование - явление отклонения направления распространения траектории света от прямолинейной траектории в гравитационном поле массивных объектов. Это достаточно хорошо изученное явление служит, в том числе, одним из независимых способов измерения некоторых космологических констант, в частности, постоянной Хаббла H0. 

Типичная гравитационно-линзированная система показана на Рисунке 1. Используется приближение плоских линз: характерные размеры самого большого объекта, который может быть линзой, - скопления галактик - порядка 1 МПк, в то время как расстояние между объектами системы порядка 100-1000 МПк. 
Гравитационная линза, подобно обычной оптической линзе, также преломляет свет, и для неё также можно составить уравнение линзы:

=-()

Все величины, входящие в формулы, должны быть описаны. Например, “где \beta - это то-то, \theta - это то-то...” или “где углы такие-то обозначены на Рисунке таком-то”. Это касается абсолютно всех формул в статье. 
Поскольку возникает необходимость свести описание трёхмерного объекта как двумерного, вводится понятие поверхностной плотности:
()=(,z)dz
а также конвергенции (convergence):
, 
Критическое значение поверхностной плотности - порядка 1гр/см^2. Также вводится так называемый радиус Эйнштейна:

Это радиус такой окружности, что поверхностная плотность заключённой внутри неё массы равна критической. Радиус Эйнштейна играет роль характерного размера в плоскости линзы.

Усиление и искажение изображений

С точки зрения линейной алгебры, гравитационное линзирование - это отображение одной плоскости на другую, которое задаётся следующей матрицей:

В ней есть две основные составляющие: конвергенция, отвечающая за изменение линейных размеров изображения (увеличение или уменьшение), и двухкомпонентный сдвиг (shear), характеризующий искажение формы изображений. Усиление изображения обратно пропорционально определителю этой матрицы (то есть якобиану этого отображения):

Видно, что существует множество точек, в которых усиление формально бесконечно. Такие точки в плоскости линзы называются критическими кривыми, а их прообразы в плоскости источника - каустиками. На Рисунке 2 видно, как ведёт себя изображение (в плоскости линзы) источника при пересечении им складки (fold) или излома (cusp) каустики.

Появление нескольких изображений

Важным свойством гравитационного линзирования является возможность появления нескольких изображений одного и того же источника. Распространяясь от источника к наблюдателю, свет отклоняется. Это отклонение вызывает задержку между временем излучения фотона из источника и моментом прихода света к наблюдателю (относительно ситуации, когда линза отсутствует). У этой задержки есть две компоненты: геометрическая и гравитационная:


Первая получается из чисто геометрических соображений - из-за разной длины путей отклонённых лучей по сравнению с невозмущёнными (пропорциональна квадрату углового расстояния между настоящей позицией источника и его изображением). Вторая следует из замедления фотонов, распространяющихся в гравитационном поле линзы, и поэтому связана с линзирующим потенциалом (эффект Шапиро). Согласно принципу Ферма, изображения формируются там, где градиент t равен нулю, следовательно, возможно появление нескольких изображений одного и того же источника. На Рисунке 3 изображена ситуация, в которой у одного источника наблюдается 6 изображений.

В 1964 году Рефсдал предложил использовать временную задержку между появлениями одного и того же изображения -??? для измерения постоянной Хаббла. Оба слагаемых и полная временная задержка пропорциональны H_0^(-1).

Возможные источники (?)

Для таких измерений блеск источника должен быть переменным. В таком случае, разные изображения, возникающие в результате гравитационного линзирования будут изменять свой блеск также, как и источник, но с некоторой задержкой по времени. В качестве возможных источников для таких измерений используются квазары - они весьма яркие и почти точечные. Однако переменность из блеска обусловлена стохастическими процессами, что может вносить существенную систематическую ошибку. Тем не менее, на текущий момент опубликовано большое количество работ, посвящённых гравитационно линзированным квазарам и измерению при их помощи постоянной Хаббла.

Другим, более стабильным источником являются сверхновые - их кривые блеска имеют чётко выраженный пик, а наблюдения занимают сравнительно небольшие времена, что значительно упрощает измерения. Кроме того, для сверхновых типа Ia - стандартных свечей - можно оценить болометрическую (интегральную по всему спектру) светимость из независимых соображений, а значит, получить абсолютное значение усиления светового потока. Эта информация, недоступная для линзированных квазаров, позволяет уменьшить количество свободных параметров модели линзы и снять определенные вырождения, а значит, уменьшить ошибки на  оцениваемые космологические параметры.  

На текущий момент известны только две гравитационно линзированные сверхновые с множественными (и при этом разрешёнными) изображениями, SN Refsdal и SN iPTF16geu. Однако с запуском в ближайшее время обсерватории LSST ожидается открытие десятков таких систем, что делает задачу разработки алгоритма анализа гравитационно линзированных сверхновых важной и своевременной.

SN Refsdal

SN Refsdal находится в рукаве спиральной галактики на z_S=1.49 и линзируется скоплением галактик кластером MACSJ1149.6+2223, находящимся на z_L=0.54, таким образом, что наблюдаются сразу три изображения родительской спиральное галактики (изображения 1.1, 1.2 и 1.3 на правой панели Рисунка 4). При этом в изображении 1.1 сверхновая дополнительно линзируется эллиптической галактикой-членом скопления таким образом, что формируются четыре её изображения S1-S4 (правая нижняя панель Рисунка 4), расположенных в виде “креста Эйнштейна”. Изображение сверхновой SX (средняя панель справа) интересно тем, что его появление в 2015 году было предсказано с высокой точностью - Келли и др. 2014. Согласно теоретическим оценкам, SY (правая верхняя панель) - это изображение сверхновой, которое “вспыхнуло” 17 лет назад (ССЫЛКА) и уже угасло.


    \backmatter

    \printbib
    
    \end{document}
