Как обсуждалось выше, микролинзирование может вносить существенный вклад в наблюдаемую кривую блеска, если размер источника не превышает радиус Эйнштейна $theta_E$ для звезды-микролинзы. Для конфигурации SN Refsdal (см. Рис. \ref{fig:snrefsdalfig}) оценим $\theta_E$ для звезды с массой $1 M_{\odot}$. Расстояния до галактики-линзы (члена скопления MACSJ1149.6+2223, $z_L=0.54$), до источника ($z_S=1.49$), а также между линзой и источником равны, согласно формуле \eqref{ang_dia_dist}, 
$$ D_d=D_A(0,z_L)=1311.54 \ \textrm{МПк}, $$
$$ D_s=D_A(0,z_S)=1744.81 \ \textrm{МПк}, $$
$$ D_{ds}=D_A(z_L,z_S)=932.47 \ \textrm{МПк}, $$
соответственно. В результате, радиус Эйнштейна для звезды с массой $1 M_{\odot}$ составляет, согласно формуле \eqref{r_ein}, $\theta_E = 2 \cdot 10^{-6}$ угловых секунд.

Для сверхновых II типа максимальный размер фотосферы составляет $R_{SN} \sim 10^{15}$ см (\cite{razmer}), или, в угловых единицах, $\theta_{SN} \sim 3 \cdot 10^{-7}$ угловых секунд. Таким образом, угловой размер SN Refsdal не превышает характерный радиус Эйнштейна, 
а значит, микролинзирование может вносить существенные искажения в её кривые блеска.