Как обсуждалось выше, микролинзирование может вносить существенный вклад в наблюдаемую кривую блеска, если размер источника не превышает радиус Эйнштейна для звезды-микролинзы. Для конфигурации SN Refsdal оценим радиус Эйнштейна для звезды с массой $1 M_{\odot}$. Расстояния (углового диаметра) в системе определяются по следующей формуле (см, напр., уравн. №№14,15 в \cite{distance_measures}):

\begin{equation}
D_{A}\left(z_{1}, z_{2}\right)=\frac{c}{1+z_{2}} \int_{z_{1}}^{z_{2}} \frac{d z}{H_{0} \sqrt{\Omega_{m}\left(1+z^{3}\right)+\Omega_{\Lambda}}}
\end{equation}

где $H_0=70$ (км/с)/МПк, $\Omega_m=0.3, \Omega_\Lambda=0.7$. Расстояния до галактики-линзы (члена скопления MACSJ1149.6+2223, $z_L=0.54$), до источника ($z_S=1.49$), а также между линзой и источником равны 

$D_d=D_A(0,z_L)=1311.54 МПк, \\
D_s=D_A(0,z_S)=1744.81 МПк,  \\
D_ds=D_A(z_L,z_S)=932.47 МПк, $

соответственно. В результате, радиус Эйнштейна для звезды с массой $1 M_{\odot}$ составляет, согласно формуле \eqref{r_ein}, $\theta_E = 2 \cdot 10^{-6}$ угловых секунд.

Для сверхновых II типа максимальный размер фотосферы составляет $R_{SN} \sim 10^{15}$ см \cite{razmer}, или, в угловых единицах, $\theta_{SN} \sim 3 \cdot 10^{-7}$ угловых секунд. Таким образом, её размер не превышает характерный радиус Эйнштейна, что делает микролинзирование одним из возможных и еще неучтенных источников ошибок при определении постоянной Хаббла.
