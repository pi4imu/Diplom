Как известно, микролинзирование при больших размерах источника вносит несущественный вклад в блеск источника (\cite{schneider1992}). Для того, чтобы оценить размер источника, начиная с которого вкладом микролинзирования можно пренебречь, мы оценили величину стандартного отклонения микро-усилений в звездных величинах $\delta m_{obs}$ как функцию размера источника. 
Для этой цели было сгенерировано 10 различных карт микрокаустик, каждая со следующими параметрами: $\kappa_*=0.4, \kappa_c = 0$, размер - $ 150 \times 150 $ радиусов Эйнштейна, разрешение - $1000 \times 1000$ пикселей. Источник моделировался кругом с постоянной поверхностной яркостью. Для различных значений радиусов источника производилась операция свёртки \textit{(convolution)} с каждой из построенных карт, в результате чего на выходе получалась новая карта с учётом неточечности источника. Далее, для каждого значения радиуса источника вычислялось стандартное отклонение  $\delta m_{o b s}$ по следующей формуле: 
\begin{equation}
\delta m_{o b s}=\sqrt{\frac{1}{N} \sum_{i=1}^{N}\left(x_{i}-\overline{x}\right)^{2}},
\end{equation}
где N - количество элементов в выборке, полученной объединением всех карт, $x_i$ - значения усилений, $\overline x$- среднее значение усиления по выборке. Полученная зависимость приведена на Рисунке \ref{fig:refstab}. 

\begin{figure}[H]
    \centering
	\includegraphics[scale=0.69]{pics/size_np_std.png}
	\caption{Зависимость стандартного отклонения $\Delta m_{obs}$ от среднего значения усиления. Пунктирной линией показана теоретическая оценка $\delta m (\theta_s)$ для источников с $\theta_s > 5$ (\cite{refsdalstabell1991}). Плотность звёзд: $\kappa_*=0.4$. Плотность тёмной материи и внешний сдвиг: $\kappa_c=\gamma=0$. \label{fig:refstab} } 
\end{figure}
Пунктирной линией показана теоретическая оценка стандартного отклонения микро-усилений для больших источников (угловой размер которых превышает 5 радиусов Эйнштейна) из работы (\cite{refsdalstabell1991}):
\begin{equation}
\delta m_{o b s} \approx \frac{2.17 \sqrt{|\kappa_*|}}{\theta s}
\end{equation}
где $\kappa_*$ - безразмерная поверхностная плотность звёзд, $\theta_s$ - размер источника в единицах радиуса Эйнштейна для характерной массы звезды (здесь -- для $1 M_{\odot} $). Формула предполагается справедливой при $\gamma=0, \kappa_* < 1$. 
Видно, что при увеличении размера источника на порядок $\delta m_{obs}$ уменьшается примерно так же на порядок, из чего можно сделать вывод, что для больших источников микролинзирование несущественно.