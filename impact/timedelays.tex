Для каждого изображения вклад микролинзирования уникален, не зависит от других изображений, что вносит некоторую неопределенность в определение временных задержек между изображениями. 

Для количественных оценок точности определения $\Delta t$ между двумя кривыми блеска с учетом микролинзирования использовался подход, детально описанный в работе \cite{doblerkeeton2006} -- пионерской работе по этому направлению. За основу была взята одна из кривых блеска SN Refsdal, рассчитанных в гидродинамической модели в различных частотных фильтрах для 400 дней с момента её взрыва (Бакланов и др., в подготовке). Она изображена на Рисунке \ref{fig:lightcurves}. 

\begin{figure}[H]
    \centering
	\includegraphics[scale=0.72]{pics/lightcurves.png}
	\caption{Кривые блеска SN Refsdal в различных фильтрах, полученные в гидродинамической модели (Бакланов и др., в подготовке). \label{fig:lightcurves}} 
\end{figure}

Для простоты анализа рассматривалась кривая блеска только в одном фильтре - F160W. Для иллюстрации влияния микролинзирования использовались карты для областей изображений S1 и S2 (см. Рис. \ref{fig:s1s4}). SN Refsdal моделируется расширяющимся со скоростью 5000 км/с кругом с постоянной поверхностной яркостью.

Предполагалось, что оригинальная кривая блеска наблюдается в изображении S1. Для имитации изображения S2 имеющаяся кривая блеска дублировалась и сдвигалась по времени (далее эта разница обозначается как истинная временная задержка $\Delta t_{\textrm{ист.}}$) и звёздной величине относительно оригинальной. После этого к каждой из кривых блеска добавлялись уникальные “шумы”, вызванные микролинзированием, полученные при помощи программного пакета {\tt{SNTD}}. Для полученных “зашумленных” кривых блеска временная задержка между изображениями определялась путем минимизации следующего функционала (аналогично работе \cite{doblerkeeton2006}:

\begin{equation}\label{chi2}
\chi^{2}(\Delta t)=\frac{1}{N} \sum_{i=1}^{N} \frac{1}{\sigma_{i}^{2}}\left[D^{+}\left(t_{i}\right)-D^{-}\left(t_{i}-\Delta t\right)-k\right]^{2}
\end{equation}
где $D^+$ и $D^-$ - значения кривых блеска, выраженные в звёздных величинах, $\sigma_i$ - фотометрическая погрешность с нормальным распределением (она предполагается постоянной во времени), $k$ - нормировочная постоянная, связанная с различным макроусилением псевдо-изображений. Далее эта операция многократно повторялась для 

Результаты представлены в следующей секции.
