Для каждого изображения вклад микролинзирования уникален, не зависит от других изображений, что вносит некоторую неопределенность в определение временных задержек между изображениями. В работе \cite{doblerkeeton2006} впервые получены количественные оценки на точность определения $\Delta t$ между двумя кривыми блеска с учетом микролинзирования. Авторы этой работы моделировали кривые блеска линзированных сверхновых следующим образом: кривая блеска некоторой реально наблюдавшейся (не линзированной) сверхновой дублировалась и “сдвигалась” по времени и звездной величине относительно настоящей, имитируя, таким образом, два различных изображения одного и того же источника. После этого к каждой из кривых блеска был добавлены “шумы”, вызванные микролинзированием. Для полученных “зашумленных” кривых блеска временная задержка между изображениями определялась путем минимизации следующего функционала:

\begin{equation}\label{chi2}
\chi^{2}(\Delta t)=\frac{1}{N} \sum_{i=1}^{N} \frac{1}{\sigma_{i}^{2}}\left[D^{+}\left(t_{i}\right)-D^{-}\left(t_{i}-\Delta t\right)-k\right]^{2}
\end{equation}
где $D^+$ и $D^-$ - значения кривых блеска, выраженные в звёздных величинах, $\sigma_i$ - фотометрическая погрешность с нормальным распределением (она предполагается постоянной во времени), k - нормировочная постоянная, связанная с различным макроусилением псевдо-изображений. 

В данной работе используется схожий подход. За основу взята кривая блеска SN Refsdal, рассчитанная в гидродинамической модели для 400 дней с момента её взрыва (Бакланов и др., в подготовке). Для иллюстрации влияния микролинзирования рассматриваются области изображений S1 и S2. SN Refsdal моделируется расширяющимся со скоростью 5000 км/с кругом с постоянной поверхностной яркостью.