При анализе гравитационно линзированных сверхновых с наблюдаемыми множественными изображениями одним из важнейших систематических эффектов является микролинзирование звездами галактики-линзы, которое может существенным образом изменять наблюдаемые кривые блеска. Звезды в галактике образуют богатую сеть каустик, что приводит к тому, что наблюдаемые кривые блеска при расширении сверхновых испытывают зависящие от времени усиления/ослабления, уникальные для каждого изображения.

Для иллюстрации влияния микролинзирования на усиление изображений точечного источника в данной работы приведены карты микрокаустик, показывающие величину усиления или ослабления источника вследствие только эффекта микролинзирования, а также построены плотности вероятности микроусилений. Как и ожидалось, при уменьшении количества звезд-микролинз при сохранении полной поверхностной плотности дисперсия плотности вероятности микролинзирования уменьшается.

Для оценки размера источника, начиная с которого вкладом микролинзирования можно пренебречь, была численно получена зависимость стандартного отклонения микро-усилений от размера источника. При этом источник моделировался кругом с постоянной поверхностной яркостью. Полученная зависимость хорошо согласуется с теоретическим предсказанием из работы [11]. 

Было показано, что для SN Refsdal микролинзирование учитывать необходимо, т.к. типичный угловой размер фотосферы сверхновой SN310-7угловых секунд не превышает радиуса Эйнштейна для звезд-микролинз массой, равной солнечной: E=210-6 угловых секунд.

Как показывают оценки, приведённые в работе [12], в результате влияния микролинзирования точность измерения временных задержек между изображениями гравитационно линзированной сверхновой ограничивается величиной порядка нескольких дней.

В дальнейшем мы планируем провести моделирование влияния микролинзирования, учитывая реалистичное распределение яркости SN Refsdal, а также протяженность источника излучения, и оценить погрешность, вносимую микролинзированием, в опредение временных запаздываний между изображениями SN Refsdal и, как следствие, на точность определения постоянной Хаббла .
