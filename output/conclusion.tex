При анализе гравитационно линзированных сверхновых с наблюдаемыми множественными изображениями одним из важнейших систематических эффектов является микролинзирование звездами галактики-линзы, которое может существенным образом изменять наблюдаемые кривые блеска. Звезды в галактике образуют богатую сеть каустик, что приводит к тому, что наблюдаемые кривые блеска при расширении сверхновых испытывают зависящие от времени усиления/ослабления, уникальные для каждого изображения.

Для иллюстрации влияния микролинзирования на усиление изображений точечного источника в данной работы приведены карты микрокаустик, показывающие величину усиления или ослабления источника вследствие только эффекта микролинзирования, а также построены плотности вероятности микроусилений. Как и ожидалось, при уменьшении количества звезд-микролинз при сохранении полной поверхностной плотности дисперсия плотности вероятности микролинзирования уменьшается.

Для оценки размера источника, начиная с которого вкладом микролинзирования можно пренебречь, была численно получена зависимость стандартного отклонения микро-усилений от размера источника. При этом источник моделировался кругом с постоянной поверхностной яркостью. Полученная зависимость хорошо согласуется с теоретическим предсказанием из работы \cite{refsdalstabell1991}. 

Было показано, что для SN Refsdal микролинзирование учитывать необходимо, т.к. типичный угловой размер фотосферы сверхновой $\theta_{SN} \sim 3 \cdot 10^{-7}$ угловых секунд не превышает радиус Эйнштейна для звезд-микролинз массой, равной $1 M_{\odot}$: $\theta_E \sim 2 \cdot 10^{-6}$ угловых секунд.

Была проанализирована зависимость дисперсии временной задержки от момента начала наблюдения её кривой блеска. Показано, что в результате влияния микролинзирования точность измерения временных задержек между изображениями гравитационно линзированной сверхновой ограничивается величиной порядка нескольких дней, что хорошо согласуется с результатами работы \cite{doblerkeeton2006} -- пионерской работы в этом направлении.

Результаты этой работы предполагается использовать для дальнейшего количественного и качественного обоснования необходимости учёта микролинзирования как одного из возможных источников ошибок при определении постоянной Хаббла в исследованиях гравитационно линзированных сверхновых.
