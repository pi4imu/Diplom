В настоящее время значения основных космологических параметров известны с очень высокой точностью. Однако недавно было обнаружено расхождение на уровне значимости примерно 3$\sigma$ величины постоянной Хаббла, определяющей темп расширения Вселенной в современную эпоху. Для понимания причин этого расхождения необходимо привлечение независимых подходов, способных также с высокой точностью определять фундаментальные космологические параметры. Одной из таких возможностей является использование наблюдений гравитационно линзированных систем, в частности, гравитационно линзированных сверхновых. Точность оценки постоянной Хаббла из наблюдений таких систем напрямую зависит от точности определения временных запаздываний между изображениями источника. Существующие в настоящее время методы анализа кривых блеска линзированных сверхновых не учитывают влияние эффекта гравитационного линзирования на отдельных звездах (микролинзирование), попавших в “конус” зрения, хотя звезды в галактике образуют богатую сеть каустик, что приводит к тому, что наблюдаемые кривые блеска при расширении сверхновых испытывают зависящие от времени усиления или ослабления, уникальные для каждого изображения.
