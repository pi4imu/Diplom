Данная работа посвящена изучению влияния микролинзирования на кривые блеска линзированных сверхновых на примере SN Refsdal — первой обнаруженной гравитационно линзированной сверхновой со множественными изображениями. Для различных параметров, характеризующих галактику-линзу, получена большая выборка карт усилений, возникающих вследствие только эффекта микролинзирования, и проведено их статистическое исследование. Построены распределения вероятности усиления в звездных величинах, изучено влияние микролинзирования на кривые блеска сверхновой в рамках модели расширяющегося диска с постоянной поверхностной яркостью.
